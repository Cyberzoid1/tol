% file contaiing a list of descriptions for potential improvements to the GUI
% these features are not core to the program, but could be nice additions as time allows
% Matthew

% Matthew Holman			10-29-2016
% cs383 - Assignment #4 - spec document portion

\documentclass[./spec.tex]{subfiles}

\begin{document}
\section{Additional Features/ Potential Improvements}
{There are a variety of additional features and potential improvements that we believe can be made to the basic functionality demonstrated in both versions 1 and 2 of the tower lights design program. The first of the following improvements are ranked first by the number of times they were mentioned in the design meeting, with the most often mentioned features ranked first and descending from there. Please note that after the first few entries, the features are essentially unranked as they were each only mentioned once and thus have equal weighting in the list.}

{\begin{itemize}
	\item Clear Frame (Four Mentions)
	\begin{itemize}
		\item The user should be able to clear the contents of a frame either by clicking a button, pressing the `del' key, etc.
	\end{itemize}
	\item Multi-Selection/ Editing (Three Mentions)
	\begin{itemize}
		\item The user should be able to select multiple frames, perhaps click + shift key, banding box, etc. and then edit those frames by changing their location in the animation, clearing their contents, etc.
	\end{itemize}
	\item Predefined Animations (Three Mentions)
	\begin{itemize}
		\item A series of predefined animations (comets, raindrops, snake effect, text, etc) should be available to insert into the final set of frames and will be able to last a specified duration.
	\end{itemize}
	\item Random Frame Generator/ Animation Generator (Three Mentions)
	\begin{itemize}
		\item With the click of a button, the user should be able to generate random color values in either a single frame, or over multiple frames.
	\end{itemize}
	\item Playback Preview (Two Mentions)
	\begin{itemize}
		\item A dedicated view/ window displays the playback of the frames as it would appear on the actual tower. This feature could possibly include playback of the music as well, in conjunction with the changing color values so that the entire effect can be previewed.
	\end{itemize}
	\item Adjust Position of Frames (Two Mentions)
	\begin{itemize}
		\item The user should be able to move an existing frame from one part of the animation to another without having to redraw the entire frame from scratch.
	\end{itemize}
	\item Auto-Import Time Codes Based on Music (Two Mentions)
	\begin{itemize}
		\item When a user selects a song, generate a number of empty frames in the animation based on the length/ tempo of the song.
	\end{itemize}
	\item Window Scaling (One Mention)
	\begin{itemize}
		\item When the user resizes the window, perhaps to a fullscreen view, the various tools/ views, etc. should scale appropriately.
	\end{itemize}
	\item Time Units (One Mention)
	\begin{itemize}
		\item Time codes for frames can be generated based on the particular tempo of a song for a set duration. For example, one section could have one frame per beat for sixteen beats at a speed of 120 beats per minute, then another section immediately following that one might have twenty beats at a speed of 136 beats per minute.
	\end{itemize}
	\item Customizable Defaults (One Mention)
	\begin{itemize}
		\item If there is a particular workflow/ layout/ set of color choices etc. that a user uses, those choices should be able to be saved and used again for a later project.
	\end{itemize}
	\item Scrollable Context (One Mention)
	\begin{itemize}
		\item The frames of the animation should be visible and the user should be able to scroll through them all.
	\end{itemize}
	\item Save Custom Colors (One Mention)
	\begin{itemize}
		\item The user should be able to save a specified color to a color palette for use later in the animation process.
	\end{itemize}
	\item Move Tower Indicator Box/ Snapshot Button $\rightarrow$ Frame (One Mention)
	\begin{itemize}
		\item Currently, in the second version of the tower of lights project, there is a design window where the user can enter information outside a `view frame' that will be used to generate the individual frames of animation. Currently, the view frame is static and the information around it moves in relation to the static frame. This leads to information loss when the user moves the information too far to the edges. By moving the view frame and having the user click a `snapshot' button to confirm that data that should be written to the animation frame, this data loss is mitigated.
	\end{itemize}
	\item Arrow Keys $\rightarrow$ Duplicate Frame (One Mention)
	\begin{itemize}
		\item Similar to how the directional buttons on the second version of the tower of lights project function, now the user should be able to use the arrow keys to get the same functionality.
	\end{itemize}
	\item Shortcuts to Adjust LED Settings (One Mention)
	\begin{itemize}
		\item The color/ brightness of the LEDs should be adjustable quickly and easily.
	\end{itemize}
	\item Zoom View of Design Panel (One Mention)
	\begin{itemize}
		\item The user should be able to zoom in and out of the design panel, similar to the one implemented in version two of the original tower of lights project, to see more or less of the surrounding information that the frame can utilize for animations. 
	\end{itemize}
	\item Auto-Save User Progress (One Mention)
	\begin{itemize}
		\item Automatically save the users work every so often - just in case the program crashes.
	\end{itemize}
\end{itemize}}
\end{document}
